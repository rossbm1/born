\documentclass[a4paper, 12pt]{article}
\usepackage{biblatex}
\iffalse
\addbibresource{s1.bib}
\addbibresource{s3.bib}
\addbibresource{s4.bib}
\addbibresource{s5.bib}
\fi
\addbibresource{sources.bib}
\RequirePackage{rosshead2}


\newcommand{\myname}{Ross McNeill}
\newcommand{\mymail}{ross@anibla.net}
\newcommand{\myclass}{Born and Inverse Born Series for a Special Case of the Second Harmonic Generation Problem} %Enter class name here
%\newcommand{\lecnum}{2} % Enter lecture number here
\newcommand{\hh}[1]{\rule{\linewidth}{#1}} % Create horizontal rule command with 1
\newcommand{\hwnum}{3} %Enter class name here

\ihead{} % Page header left
\ohead{\thepage} % Page header right
\ifoot{\myname} % left footer
\ofoot{\mymail} % right footer

\title{
\normalfont \normalsize
\textsc{Drexel University, Department of Mathematics} \\ [8pt] % Your university, school and/or department name(s)
%\horrule{0.5pt} \\[0.4cm] % Thin top horizontal rule
\huge\myclass\\
}
\author{\myname}

\begin{document}
\maketitle
%\tableofcontents
%\newpage
\abstract
We study the Born and Inverse Born series for a special case of the second harmonic generation system of PDEs.
We give a recursive formula for the forward operators and prove boundedness conditions that guarantee the convergence of the Born and inverse Born series.
We also use fixed point theory to give explicit conditions for convergence of the Born series.

\section{Introduction: The Forward Problem}\label{intro}
Let $\Omega \subseteq \mathbb{R}^{d}$ with smooth boundary.
The second harmonic generation problem is to find a unique solution to the coupled system
%TODO change \eta_{2} to \eta for the WHOLE PAPER
\begin{align*}
	\begin{split}
	\Delta u^{(1)} + k^2\left( 1 + 4\pi \eta \right) u^{(1)} &= -4\pi k^2\left( \eta_1u^{(1)} + 2\eta_2u^{(2)}\left(u^{(1)}\right)^{*} \right)\\
	\Delta u^{(2)} + \left( 2k \right)^2\left( 1 + 4\pi \eta \right) u^{(2)} &= -4\pi\left( 2k \right)^2\left( \eta_1u^{(2)} + \eta_2\left(u^{(1)}\right)^{2} \right)
	\end{split}
	\qquad\text{in $\Omega$}\\
	\dfrac{\pp{u^{(1)}}}{\pp{\nu}} &= g, \qquad \dfrac{\pp{u^{(2)}}}{\pp{\nu}}= 0 \qquad \text{on } \pp{\Omega}
\end{align*}
where $\eta$ is a known constant and $\nu$ is the unit outward normal vector to $\pp{\Omega}$.
In this paper, we analyze the Born and inverse Born series and prove the existence and uniqueness of a special case of the second harmonic generation problem given by

%TODO move fixed point stuff into the existence uniqueness section.
\begin{align}\label{eq:1:1}
	\begin{split}
	\Delta u^{(1)} + k^2 u^{(1)} &= -4\pi k^2\left( 2\eta_2 u^{(2)}u^{(1)} \right) \\
	\Delta u^{(2)} + \left( 2k \right)^2 u^{(2)} &= -4\pi\left( 2k \right)^2\left( \eta_2 \left(u^{(1)}\right)^2 \right)\\
	\dfrac{\pp{u^{(1)}}}{\pp{\nu}} &= g, \qquad \dfrac{\pp{u^{(2)}}}{\pp{\nu}}= 0 \qquad \text{on } \pp{\Omega}
	\end{split}
\end{align}
Since $\eta=0$ and $\eta_1= 0$, we will abuse notation and denote $\eta \coloneqq \eta_2$
for the rest of the paper.

In order to prove the existence and uniqueness of solutions to \cref{eq:1:1}, we convert the system of PDEs into a transformation between vector spaces.
Let $G\left( k; x, y \right) $ be the Green's function for the operator $\Delta u^{(1)} + k^2$
and
$G\left( 2k; x, y \right) $ be the Green's function for the operator $\Delta u^{(1)} + \left( 2k \right)^2$.
Then we can transform the system of PDEs in \cref{eq:1:1} into an operator
\[
	T\colon C\left( \overline{\Omega} \right)^2\to C\left( \overline{\Omega} \right)^2
\]
given by
\begin{equation}\label{intro:2}
T\begin{pmatrix} u^{(1)}\\u^{(2)} \end{pmatrix}  = \begin{pmatrix} u_0^{(1)}\\u_0^{(2)} \end{pmatrix} +
\begin{pmatrix}
	-8\pi k^2 \int_{\Omega}^{} G\left(k; x,y \right) \eta\left( y \right) u^{(1)}\left( y \right) u^{(2)}\left( y \right)\dd{y}\\
	-4 \pi \left( 2k \right) ^2\int_{\Omega}^{} G\left( 2k; x,y \right) \eta\left( y \right) \left(u^{(1)}\right)^2\left( y \right)\dd{y}
\end{pmatrix}
.\end{equation}

Note that a fixed point $u \in C\left( \overline{\Omega} \right)^2$ satisfies the PDE \cref{eq:1:1}.
The following rest gives conditions for the existence of such a $u$.

\begin{thm}
	Let $T: C\left( \overline{\Omega} \right)^2 \to C\left( \overline{\Omega} \right)^2$ be defined by \cref{intro:2}.
	Define
\begin{align*}
	\mu_1 &\coloneqq 8\pi k^2 \sup_{x \in \Omega}\int_{\Omega}^{} \left| G\left(k; x,y \right) \right|  \dd{y}\\
	\mu_{2}&\coloneqq 16\pi k^2 \sup_{x \in \Omega}\int_{\Omega}^{} \left| G\left(2k; x,y \right) \right|  \dd{y}
\end{align*}
and $\mu = \max \left\{ \mu_1,\mu_2 \right\}$.
Then if
\[
\lVert \eta \rVert \le \frac{1}{4\mu \lVert u_0 \rVert}
,\] 
$T$ has a unique fixed point if the ball of radius $\lVert u_0 \rVert$ about $u_0 \in \left( C \overline{\Omega} \right)^2$.
\end{thm}
The proof is given in \Cref{sec2}

\section{The Forward Born Series}

We wish to compute the function $u$ on $\pp{\Omega}$ given a source $g $ on $\pp{\Omega}$.
Our presentation follows that in \cite{DeFilippis_2023}.
The solution to the forward problem is derived by iteration of the integral equation \cref{intro:2}. We seek a series representation of $u$ of the form
\begin{equation}\label{born1}
u = u_0 + K_1\left( \eta \right) + K_2\left( \eta,\eta \right) +
K_3\left( \eta,\eta,\eta \right) +\cdots
.\end{equation}
The forward operators are of the form
\[
	K_{n}\left( \eta,\ldots,\eta \right)  \coloneqq \begin{pmatrix} K_n^{(1)}\left( \eta,\ldots,\eta \right) \\K_n^{(2)}\left( \eta,\ldots,\eta \right)  \end{pmatrix} \colon \left[ L^{\infty}\left( \Omega \right)  \right]^{n} \to C\left( \pp{\Omega}\times \pp{\Omega} \right)^2
.\]
Now, $K_{n}$ is $n$-linear on $\left[ L^{\infty}\left( \Omega \right) \right]^{n}$.
The fixed point iteration gives:
 \[
u^{(1)}\left( x \right)\coloneqq T\left( u_0 \right)\left( x \right)
= \begin{pmatrix} u_0^{(1)}\\u_0^{(2)} \end{pmatrix}
\begin{pmatrix}
	-8\pi k^2\int_{\Omega}^{} G\left(k; x,y \right) \eta\left( y \right) u_0^{(1)}\left( y \right) u^{(2)}_0\left( y \right)\dd{y}\\
	-4 \pi \left( 2k \right)^2\int_{\Omega}^{} G\left(2k; x,y \right) \eta\left( y \right)  u_0^{(1)}u_0^{(1)}\left( y \right)\dd{y}
\end{pmatrix}
\]
which implies that
\[
K_1\left( \eta \right) \left( x \right)
=\begin{pmatrix}
	-8\pi k^2\int_{\Omega}^{} G\left(k; x,y \right) \eta\left( y \right) u_0^{(1)}\left( y \right) u^{(2)}_0\left( y \right)\dd{y}\\
	-4 \pi\left( 2k \right)^2\int_{\Omega}^{} G\left(2k; x,y \right) \eta\left( y \right)  u_0^{(1)}u_0^{(1)}\left( y \right)\dd{y}
\end{pmatrix}
.\]
To make our lives easier, define
 \begin{align*}
	h_1\left( \begin{pmatrix} v^{(1)}\\v^{(2)} \end{pmatrix} , \eta \right)
	&= -8\pi k^2\int_{\Omega}^{} G\left(k; x,y \right) \eta\left( y \right) v^{(1)}\left( y \right) v^{(2)}\left( y \right)\dd{y},\\
	h_2\left( \begin{pmatrix} v^{(1)}\\v^{(2)} \end{pmatrix} , \eta \right)
	&= -4 \pi \left( 2k \right)^2\int_{\Omega}^{} G\left(2k; x,y \right) \eta\left( y \right) v^{(1)}\left( y \right) v^{(1)}\left( y \right)\dd{y}
.\end{align*}

%TODO define the tensor product.
\begin{defn}
	Let $T_{i}$ and $T_{j}$ be multilinear operators of order $i$ and $j$ respectively.
	Define the tensor product $T_{i}\otimes T_{j}$ to be
	\[
	T_{i}\otimes T_{j}\left( \alpha_1,\ldots,\alpha_{i},\alpha_{i + 1}, \ldots, \alpha_{i + j} \right)
	= T_{i}\left(\alpha_1,\ldots,\alpha_{i}\right)  T_{i}\left(\alpha_1,\ldots,\alpha_{j}\right) 
	,\] 
	which is of order $i + j$.
\end{defn}

\begin{defn}\label{H1H2}
	%TODO make clear that T_{k} is a VECTOR of two multilinear operators, that is $T = T_{k}^{(1)}, T_{k}^{(2)}$
	Let $T_{k}\left( \alpha_1,\alpha_2,\ldots,\alpha_{k} \right)$ be a multilinear operator of order $k$.
	Define the $k + 1$ order multilinear operators $H_1T_{k}$ and $H_2T_{k}$
	by
	\[
	H_1T_{k}\left( \alpha_1,\ldots,\alpha_{k + 1} \right) =
	h_1\left( \begin{pmatrix} T_{k}^{(1)}( \alpha_1^{(1)},\ldots,\alpha_{k}^{(1)} )\\T_{k}^{(2)} ( \alpha_1^{(2)},\ldots,\alpha_{k}^{(2)} )\end{pmatrix}  ,\alpha_{k + 1} \right)
	\]
	and
	\[
	H_2T_{k}\left( \alpha_1,\ldots,\alpha_{k + 1} \right)
	h_2\left( \begin{pmatrix} T_{k}^{(1)}( \alpha_1^{(1)},\ldots,\alpha_{k}^{(1)} )\\T_{k}^{(2)} ( \alpha_1^{(2)},\ldots,\alpha_{k}^{(2)} )\end{pmatrix}  ,\alpha_{k + 1} \right)
	.\]
\end{defn}
\noindent Thus,
\[
	T\begin{pmatrix} v^{(1)}\\v^{(2)} \end{pmatrix} = \begin{pmatrix} u_0^{(1)}\\u_0^{(2)} \end{pmatrix} + \begin{pmatrix} H_1 v^{(1)}\otimes v^{(2)} \\H_2 v^{(1)} \otimes v^{(1)}\end{pmatrix}
\]
becomes a sum of multilinear operators.

\begin{lem}\label{lemma1}
	considering the $n$th iterate as a sum of multilinear operators, we have
	\begin{equation}\label{lem1:1}
		\begin{pmatrix} u_{n + 1}^{(1)}\\u_{n + 1}^{(2)} \end{pmatrix} = \begin{pmatrix} u_n^{(1)}\\u_n^{(2)} \end{pmatrix} + \begin{pmatrix} \text{multilinear operators of order $\ge n + 1$}\\\text{multilinear operators of order $\ge n + 1$} \end{pmatrix}
	.\end{equation}
\end{lem}
\begin{proof}
	Proof by induction on $n$.
	For the base case, we have
	\[
	\begin{pmatrix} u_{1}^{(1)}\\u_{1}^{(2)} \end{pmatrix} = \begin{pmatrix} u_0^{(1)}\\u_0^{(2)} \end{pmatrix} +
	\begin{pmatrix}
		H_1 u_0^{(1)} \otimes u_0^{(2)}\\
		H_2 u_0^{(1)} \otimes u_0^{(1)}
\end{pmatrix}
	.\]
	which obviously satisfies \cref{lem1:1}.
	For the induction step, we assume that
\[
		\begin{pmatrix} u_{n}^{(1)}\\u_n^{(2)} \end{pmatrix} = \begin{pmatrix} u_{n-1}^{(1)}\\u_{n-1}^{(2)} \end{pmatrix} +  \begin{pmatrix} w_1\\w_2 \end{pmatrix}
\]
where $w_1,w_2$ are sums of multilinear operators of order at least $n$.
Then,
\begin{align*}
u_n^{(1)} \otimes u_n^{(2)}
&=\left( u_{n-1}^{(1)} + w_1 \right) \otimes \left( u_{n-1}^{(2)}+ w_2 \right)\\
&=u_{n-1}^{(1)} \otimes u_{n-1}^{(2)} + u_{n-1}^{(1)} \otimes w_2 + w_1\otimes u_{n-1}^{(2)} + w_1\otimes w_2 \\
&= u_{n-1}^{(1)}\otimes u_{n-1}^{(2)} + \text{multilinear operators of order at least $n$}
.\end{align*}
A similar calculation gives
\[
u_n^{(1)} \otimes u_n^{(2)}
= u_{n-1}^{(1)}\otimes u_{n-1}^{(1)} + \text{multilinear operators of order at least $n$}
.\]
Now, applying $H_1$ (resp. $H_2$) to an operator increases the order by 1 so any product containing a $w$

Thus, we combine our results to obtain
\begin{align*}
	\begin{pmatrix} u_{n + 1}^{(1)}\\u_{n + 1}^{(2)} \end{pmatrix}
	&= \begin{pmatrix} u_0^{(1)}\\u_0^{(2)} \end{pmatrix} +
\begin{pmatrix}
		H_1 u_n^{(1)} \otimes u_n^{(2)}\\
		H_2 u_n^{(1)} \otimes u_n^{(1)}\\
\end{pmatrix}\\
&= \begin{pmatrix} u_0^{(1)}\\u_0^{(2)} \end{pmatrix} +
\begin{pmatrix}
		H_1 u_{n-1}^{(1)} \otimes u_{n-1}^{(2)}+ \text{multilinear operators of order at least $n$}\\
		H_2 u_{n-1}^{(1)} \otimes u_{n-1}^{(1)}+ \text{multilinear operators of order at least $n$}\\
\end{pmatrix}\\
&= \begin{pmatrix} u_n^{(1)}\\u_n^{(2)} \end{pmatrix}
+ \begin{pmatrix} \text{multilinear operators of order $\ge n + 1$}\\\text{multilinear operators of order $\ge n + 1$} \end{pmatrix}
\end{align*}
Where the last equality holds because applying $H_1$ and $H_2$ to the operators $u_n^{(1)}\otimes u_n^{(2)}$ and $u_n^{(1)}\otimes u_n^{(1)}$ respectively, we increase the order of the operators by one.

giving us \cref{lem1:1}.
\end{proof}

\Cref{lemma1} shows that the fixed point iteration that generates the sequence $\left\{ u_{n} \right\} $ also generates the series of the form \cref{born1}.
 For each $K_{n}$, the operator $K_n^{(1)}$ and $K_{n}^{(2)}$ are sums of multilinear operators of order $n$.
 \Cref{lemma1} also tells us that the $n$th iterate $u_{n}$ contains all terms of degree $n$.
 From this, we obtain that
\[
\begin{pmatrix} u_n^{(1)}\\u_n^{(2)} \end{pmatrix}
= \begin{pmatrix} u_0^{(1)}\\u_0^{(2)} \end{pmatrix}
+ \sum_{i=1}^{n} \begin{pmatrix} K_i^{(1)}\left( \eta,\ldots,\eta \right)\\K_i^{(2)} \left( \eta,\ldots,\eta \right)\end{pmatrix}
+\begin{pmatrix} \text{terms of order $\ge n + 1$}\\ \text{terms of order $\ge n + 1$}\end{pmatrix}
.\]
\subsection{A General Formula for the Forward Operators}
Our fixed point iteration is given by the scheme
%TODO write out some of the forward operators expertly, maybe u_{2}
%TODO add "in genera" before third thing here
\begin{align*}
	\begin{pmatrix} u_1^{(1)}\\u_1^{(2)}\end{pmatrix}  &= \begin{pmatrix} u_0^{(1)}\\u_0^{(2)} \end{pmatrix} + \begin{pmatrix} H_1 u_0^{(1)}\otimes u_0^{(2)}\\H_2 u_0^{(1)} \otimes u_0^{(1)}\end{pmatrix} \\
	\begin{pmatrix} u_2^{(1)}\\u_2^{(2)}\end{pmatrix}  &= \begin{pmatrix} u_0^{(1)}\\u_0^{(2)} \end{pmatrix} + \begin{pmatrix} H_1 u_1^{(1)}\otimes u_1^{(2)}\\H_2 u_1^{(1)} \otimes u_1^{(1)}\end{pmatrix} \\
	\begin{pmatrix} u_{n + 1}^{(1)}\\u_{n + 1}^{(2)}\end{pmatrix} &= \begin{pmatrix} u_0^{(1)}\\u_0^{(2)} \end{pmatrix} + \begin{pmatrix} H_1 u_n^{(1)}\otimes u_n^{(2)}\\H_2 u_n^{(1)} \otimes u_n^{(1)}\end{pmatrix}
.\end{align*}
To compute an explicit form for the forward operators $K_{n}$, we define
\[
\begin{pmatrix} U_n^{(1)}\\U_n^{(2)} \end{pmatrix}
\coloneqq\begin{pmatrix} u_0^{(1)}\\u_0^{(2)} \end{pmatrix}  + \sum_{i=1}^{n} \begin{pmatrix} K_i^{(1)}\left( \eta,\ldots,\eta \right)\\K_i^{(2)} \left( \eta,\ldots,\eta \right)\end{pmatrix}
.\]
That is, $U_{n}$ is the sum of the first $n$ forward operators. Now  \Cref{lemma1} gives
\[
\begin{pmatrix} u_n^{(1)}\\u_n^{(2)} \end{pmatrix} = \begin{pmatrix} U_n^{(1)}\\U_n^{(2)} \end{pmatrix}+ \begin{pmatrix} w_1\\w_2 \end{pmatrix}
.\]
where $w_1,w_2$ are sums of multilinear operators of order $\ge n + 1$.
Now, we compute $U_{n + 1}$:
\[
\begin{pmatrix} u_{n + 1}^{(1)}\\u_{n + 1}^{(2)} \end{pmatrix}
= \begin{pmatrix} u_0^{(1)}\\u_0^{(2)} \end{pmatrix}+
\begin{pmatrix} H_1 \left(U_n^{(1)} + w_1\right)\otimes \left(U_n^{(2)} + w_2\right)\\H_2 \left(U_n^{(1)} + w_1\right) \otimes \left(U_n^{(1)} + w_1\right)\end{pmatrix}
.\]
Similarly to \Cref{lemma1}, after expanding out the tensor product, any terms containing $w_1, w_2$ may be dropped as they will be of higher order than $n + 1$.
Thus, all of the terms of $\begin{pmatrix} K_{n + 1}^{(1)}\\K_{n + 1}^{(2)} \end{pmatrix} $ will be contained in
\[
\begin{pmatrix} H_1 U_n^{(1)}\otimes U_n^{(2)}\\H_2 U_n^{(1)} \otimes U_n^{(1)}\end{pmatrix}
.\]
Now, $H_1$ and $H_2$ add one to the order, $K_{n + 1}^{(1)}$ will be the sum of all terms of the form
\[
H_1 K_i^{(1)}\otimes K_j^{(2)} \quad \text{where} \quad i + j = n
\]
 and
$K_{n + 1}^{(2)}$ will be the sum of terms of the form
\[
H_2 K_i^{(i)}\otimes K_j^{(2)}, \quad \text{where} \quad i + j = n
.\]
Thus, we obtain
 \[
	 K_{n + 1}^{(1)}= H_1\sum_{\substack{0\le i,j\le n;\\ i + j = n}} K_i^{(1)} \otimes K_j^{(2)}
	 \qquad \text{and} \qquad
	 K_{n + 1}^{(2)}= H_2\sum_{\substack{0\le i,j\le n;\\ i + j = n}} K_i^{(1)} \otimes K_j^{(1)}
\]
and we have derived the following recurrence for the forward operators:
%TODO add "in genera" before third thing here
\begin{align}\label{fwop}
	K_1 &= \begin{pmatrix} K_0^{(1)}\\K_0^{(2)} \end{pmatrix} = \begin{pmatrix} u_0^{(1)}\\u_0^{(2)}\end{pmatrix},\nonumber\\
	K_2 &= \begin{pmatrix} K_1^{(1)}\\K_1^{(2)} \end{pmatrix} = \begin{pmatrix} H_1 u_0^{(1)}\otimes u_0^{(2)}\\H_2 u_0^{(1)} \otimes u_0^{(1)}\end{pmatrix},\nonumber\\
	K_{n + 1} &= \begin{pmatrix} K_{n + 1}^{(1)}\\K_{n + 1}^{(2)} \end{pmatrix} = \begin{pmatrix} H_1 \sum\limits_{\substack{0\le i,j\le n\\i + j = n}} K_i^{(1)}\otimes K_j^{(2)}\\H_2 \sum\limits_{\substack{0\le i,j\le n\\i + j = n}}K_i^{(1)} \otimes K_j^{(1)}\end{pmatrix}
.\end{align}

\subsection{Bounds on the Forward Operators}
\begin{defn}
	Let $K$ be a multilinear operator of order $n$ on $\left[ L^{\infty}\left( \Omega \right) \right]^{n}$. Define
\[
	\left| K \right|_{\infty}
	\coloneqq \sup_{\substack{\alpha_{i} \neq 0\\ 1\le i\le n}} \frac{\lVert K\left( \alpha_1,\ldots,\alpha_{n} \right) \rVert_{C\left( \pp{\Omega}\times \pp{\Omega} \right)}}{\lVert \alpha_1 \rVert_{\infty}\cdots \lVert \alpha_{n} \rVert_{\infty}}
.\]
\end{defn}

	Note that if $T$ and $K$ is are multilinear operators of order $n$, then the triangle inequality
	\[
	\left| T + K \right|_{\infty} \le \left| T \right|_{\infty} + \left| K \right|_{\infty}
	\] 
	holds.

	Define $\mu$ as it was defined in \Cref{sec2}, that is $\mu \coloneqq \max \left\{ \mu_1,\mu_2 \right\}$ where 
\begin{align*}
	\mu_1 &\coloneqq 8\pi k^2 \sup_{x \in \Omega}\int_{\Omega}^{} \left| G\left(k; x,y \right) \right|  \dd{y},\\
	\mu_{2}&\coloneqq 8\pi k^2 \sup_{x \in \Omega}\int_{\Omega}^{} \left| G\left(2k; x,y \right) \right|  \dd{y}
.\end{align*}

\begin{lem}\label{nuLem}
	The components $K_{n}^{(1)}$ and $K_{n}^{(2)}$ of $K_{n}$ defined by \cref{fwop} are bounded multilinear operators and
	\[
	\left| K_{n}^{(1)} \right|_{\infty} \le \nu_{n}^{(1)}\mu^{n}
	\quad \text{and} \quad
	\left| K_{n}^{(2)} \right|_{\infty} \le \nu_{n}^{(2)}\mu^{n}
	,\] 
	where
	\[
	\nu_0^{(1)} = \lVert u_0^{(1)} \rVert_{C\left( \overline{\Omega}\times \delta \Omega \right)}
	\quad \text{and} \quad
	\nu_0^{(2)} = \lVert u_0^{(2)} \rVert_{C\left( \overline{\Omega}\times \delta \Omega \right)}
	,\] 
	and for each $n \ge 0$,
	\begin{align}
		\label{nu1}\nu_{n + 1}^{(1)} &= \sum_{\substack{0\le i,j\le n\\i + j = n}} \nu_{i}^{(1)} \nu_{j}^{(2)},\\
		\label{nu2}\nu_{n + 1}^{(2)} &= \sum_{\substack{0\le i,j\le n\\i + j = n}} \nu_{i}^{(1)} \nu_{j}^{(1)}
	.\end{align} 
\end{lem}
	First, note that for the product given in \Cref{H1H2}, we have
	\[
		\left| H_1 T_{k} \right|_{\infty} \le \mu \left| T_{k} \right|_{\infty}
		\quad \text{and} \quad
		\left| H_2 T_{k} \right|_{\infty} \le \mu \left| T_{k} \right|_{\infty}
	.\] 
Also, we have
\[
	\left| T_{k}\otimes T_{j} \right|_{\infty} \le \left| T_{k} \right|_{\infty}\left| T_{j} \right|_{\infty}
.\] 
\begin{proof}
	Proof by induction on $n$.
	For the base case, it's obvious that $\left| K_{0}^{(1)} \right|_{\infty} = \nu_{0}^{(1)}$ and $\left| K_{0} \right|_{\infty}= \nu_0^{(2)}$.

	For the induction step, suppose that for each $i\le n$, we have
\[
	\left| K_{i}^{(1)} \right|_{\infty} \le \nu_{i}^{(1)}\mu^{i}
	\quad \text{and} \quad
	\left| K_{i}^{(2)} \right|_{\infty} \le \nu_{i}^{(2)}\mu^{i}
	.\]
	Then, by \cref{fwop},
	\begin{align*}
		\left| K_{n + 1}^{(1)} \right|_{\infty}
		&= \left| H_1 \sum_{\substack{0\le i,j\le n\\i + j = n}} K_{k}^{(1)}\otimes K_{j}^{(2)}\right|_{\infty} \\
		&\le \mu \sum_{\substack{0\le i,j\le n\\i + j = n}} \left| K_{i}^{(1)} \right|_{\infty}\left| K_{j}^{(2)} \right|_{\infty} \\
		&\le \mu \sum_{\substack{0\le i,j\le n\\i + j = n}}\nu_{i}^{(1)}\mu^{(i)} \nu_{j}^{(2)}\mu^{(j)} \\
		&= \mu^{n + 1} \sum_{\substack{0\le i,j\le n\\i + j = n}}\nu_{i}^{(1)}\nu_{j}^{(2)}\\
		&= \mu^{n + 1}\nu_{n + 1}^{(1)}
	.\end{align*}
	A similar computation reveals that $\left| K_{n + 1}^{(2)} \right|_{\infty}\le \mu^{n + 1}\nu_{n + 1}^{(2)}$.
\end{proof}

%I don't think I wanna do this. I think I wanna replicate the generating function proof of each \nu_{k}^(i) first and then combine the results at the end.
\iffalse
Using the previous lemma, we can bound the forward operators. First, we define 
\begin{equation}\label{nu}
	\nu_{n} \coloneqq \left[\left(\nu_{n}^{(1)}\right)^2 + \left(\nu_{n}^{(2)}\right)^2 \right]^{1/2}
.\end{equation} 
\begin{cor}
	The forward operators $K_n$ are bounded multilinear operators from $\left[ L^{\infty}\left( \Omega \right) \right]^{n}$ to $C\left( \pp{\Omega}\times \pp{\Omega} \right)^2$ that satisfy
	\[
	\lVert K_{n} \rVert= \lVert \begin{pmatrix} \left| K_{n}^{(1)} \right|_{\infty}\\\left| K_{n}^{(2)} \right|_{_{\infty}} \end{pmatrix} \rVert \le \nu_{n}\mu^{n}  
	.\] 
\end{cor}
\fi

\begin{lem}
	For the sequences $\{ \nu_{n}^{(1)} \}$ and $\{ \nu_{n}^{(2)} \}$ defined in \Cref{nuLem}, there exist constants $\nu^{(1)},\nu^{(2)}$ and $\kappa^{(1)}, \kappa^{(2)}$ such that
	\begin{align}
		\label{convg1}\nu_{n}^{(1)} &\le \nu^{(1)} \left( \kappa^{(1)} \right)^{n}\\
		\label{convg2}\nu_{n}^{(2)} &\le \nu^{(2)} \left( \kappa^{(2)} \right)^{n}
	.\end{align} 
	\[
	\nu^{(1)} = \nu^{(2)} = \lVert u_0 \rVert_{C\left( \overline{\Omega} \times \pp{\Omega} \right)^2}
	\] 
	and 
	\[
		%K^{(1)} = K^{(2)} = 4\lVert u_0 \rVert_{C\left( \overline{\Omega} \times \pp{\Omega} \right)^2}
		\kappa^{(1)} = \kappa^{(2)} = 4\lVert u_0 \rVert_{C\left( \overline{\Omega} \times \pp{\Omega} \right)^2}
	.\] 
\end{lem}
\begin{proof}
	Let
	\begin{align*}
		P\left( x \right) &= \sum_{n=0}^{\infty} \nu_{n}^{(1)}x^{n} \\
		Q\left( x \right) &= \sum_{n=0}^{\infty} \nu_{n}^{(2)}x^{n}
	.\end{align*}
	Then we have that
	\[
	P\left( x \right) Q\left( x \right) = \sum_{n=0}^{\infty} \sum_{i + j = n}^{} \nu_{i}^{(1)}\nu_{j}^{(2)}x^{n}
	\] 
	and 
	\[
		\left( P\left( x \right) \right) ^2
		= \sum_{n=0}^{\infty} \sum_{i + j = n} \nu_{i}^{(1)}\nu_{j}^{(1)}x^{n}
	.\] 
Thus, we can multiply \cref{nu1,nu2} by $x^{n}$ and sum to obtain
\begin{align}
	\label{Px}\frac{P\left( x \right) -\nu_0^{(1)}}{x} &= P\left( x \right) Q\left( x \right)  \\
	\label{Qx}\frac{Q\left( x \right) -\nu_0^{(2)}}{x} &= \left(P\left( x \right)\right)^2
.\end{align}
We may transform $\cref{Qx}$ into the form $Q= x P^2 + \nu_0^{(2)}$ and plug back into \cref{Px} to obtain
\[
x^2P^3+ x\nu_0^{(2)}P- P +\nu_0^{(1)} = 0
.\] 
Differentiate with respect $x$ and obtain
\[
3x^2P^2P' + 2xP^3 + \nu_0^{(2)}P + x\nu_0^{(2)}P' - P' = 0
.\] 
Thus,
\[
P'\left(3x^2P^2+ x\nu_0^{(2)}-1\right) = -2xP^3 - \nu_0^{(2)}P
.\] 
and so
\begin{equation}\label{odefin}
P' = -\frac{2xP^3 + \nu_0^{(2)}P}{3x^2P^2 + x\nu_0^{(2)}- 1}
.\end{equation} 
Now, as long as $3x^2P^2 + x\nu_0^{(2)}- 1 \neq 0$, $P'$ is an analytic function of $x$ and $P$ which means that there is a unique analytic solution for \cref{odefin} which is exactly $P\left( x \right)$. 
Thus, the power series $P\left( x \right) $ has a positive radius of convergence.

Since $Q = xP^2+ \nu_0^{(2)}$, whenever $P\left( x \right) $ is an analytic function, so is $Q\left( x \right)$.
Thus, both $P\left( x \right)$ and $Q\left( x \right)$ have positive radii of convergence and the proof is complete.
\end{proof}

%TODO get explicit values for K^(1) and K^(2) and $\nu^(1) and \nu^(2)
%TODO switch $K^(1)$ to \kappa^(1)\ldots

\begin{prop}\label{convgprop}
	The forward operator $K_{n}$ given by \cref{fwop} is a bounded multilinear operator from $\left[ L^{\infty}\left( \Omega \right)\right]^{n} $ to $ C\left( \pp{\Omega} \right)  ^2$, and
	\begin{equation}\label{fwopbnd}
		\left| K_{n} \right| \le \nu \left( \kappa \mu \right)^{n}
	\end{equation}
	where $\kappa = \max \left\{ \kappa^{(1)}, \kappa^{(2)} \right\}$,
	$\nu = 2\max \left\{ \nu^{(1)}, \nu^{(2)} \right\}$,
	and $\mu =\max \left\{ \mu_1,\mu_2 \right\} $ for
	\[
	\mu_1 = 8\pi k^2 \sup_{x \in \Omega}\int_{\Omega}^{} \left| G\left( k; x, y \right)  \right| \dd{y}
	\qquad \text{and} \qquad
	\mu_2 = 16\pi k^2 \sup_{x \in \Omega}\int_{\Omega}^{} \left| G\left( 2k; x, y \right)  \right| \dd{y}
	\] 
	as in \cref{sec2}.
\end{prop}
\begin{proof}
	\begin{align*}
		\left| K_{n} \right|^2 &= \left| \begin{pmatrix} K_{n}^{(1)}\\K_{n}^{(2)} \end{pmatrix}  \right|^2\\
	&= \left| K_{n}^{(1)} \right|_{\infty}^2 + \left| K_{n}^{(1)}\right|_{\infty}^2\\
	&\le \left( \nu^{(1)} \left( \kappa^{(1)}\mu \right)^n \right)^2 +  \left( \nu^{(2)} \left( \kappa^{(2)}\mu \right)^n \right)^2\\
	&\le \left( \nu^{(1)} \left(\kappa^{(1)}\mu \right)^n + \nu^{(2)} \left( \kappa^{(2)}\mu \right)^n \right)^2\\
	&\le \nu^2 \left( \kappa\mu \right)^{2n}
	.\end{align*}
\end{proof}

\begin{cor}
	The Born series 
	\[
	u = u_0 + \sum_{n=1}^{\infty} K_{n}\left( \eta,\ldots,\eta \right) 
	\] 
	where $K_{n}$ are given by \cref{fwop} converges in $C\left( \overline{\Omega} \right)^2$ for
	\[
	\lVert \eta \rVert_{\infty} \le  \frac{1}{\kappa \mu}
	.\] 
\end{cor}


\section{Inverse Born Series}

The inverse problem is to reconstruct the coefficient $\eta$ from measurements on the boundary $\phi = u - u_0$ on $\pp{\Omega}$.
We define the Inverse Born Series (IBS) as 
\begin{equation}\label{ibs}
	\tilde{\eta} = \mathcal{K}_{1}\left(\phi\right) + \mathcal{K}_{2}\left( \phi \right) + \mathcal{K}_{3}\left( \phi \right) + \cdots
\end{equation}
where the data $\phi \in  C\left( \pp{\Omega} \right) ^2$.
We have already seen the IBS analyzed in \cite{Hoskins_2022, Moskow_2008}. The inverse operators $\mathcal{K}_{m}$ are given by
\begin{align*}
	\mathcal{K}_{1}\left( \phi \right) &= K_1^{+}\left( \phi \right),\\
	\mathcal{K}_{2}\left( \phi \right) &= -\mathcal{K}_1\left( K_2\left( \mathcal{K}_1\left( \phi \right) , \mathcal{K}_{1}\left( \phi \right)  \right)  \right) ,\\
	\mathcal{K}_{m}\left( \phi \right)  &=  -\sum_{n=2}^{m} \, \sum_{i_1 + \cdots + i_{n} = m} \mathcal{K}_{1}\left(K_{n}\left( \mathcal{K}_{i_{1}}\left( \phi \right) , \ldots, \mathcal{K}_{i_{n}}\left( \phi \right)  \right)\right)
.\end{align*}
Note that the operator $K_1$ does not, in general, have a bounded inverse. Thus, we take $\mathcal{K}_{1}$ to be the regularized pseudoinverse $K_{1}^{+}$ of $K_1$ as described in \cite{MoskowSchotland+2019+273+296}.
Now, our bounds on the forward operators given in \cref{convgprop} in combination with theorems 2.2 and 2.4 of \cite{Hoskins_2022} gives the next two theorems on convergence and approximation error of the IBS.
\begin{note}
	We denote by $\lVert \mathcal{K}_{1} \rVert$ the operator norm of $\mathcal{K}_{1}$ as a map from $C\left( \pp{\Omega} \right)^2$ to $L^{\infty}\left( \Omega \right)$.
\end{note}

\begin{thm}[convergence of the IBS]\label{ibsconvg}
	If $\lVert \mathcal{K_1}\left( \phi \right)  \rVert< r$ where
	\[
	r = \frac{1}{2\kappa \mu}\left( \sqrt{16 C^2 + 1} -4C \right)
	,\] 
	$C = \max \left\{ 2, \lVert \mathcal{K}_{1}\rVert \nu \kappa \mu \right\} $, and $\nu, \kappa$ are the same as in \Cref{convgprop}, then the IBS \cref{ibs} converges.
\end{thm}

\begin{thm}[approximation error] Suppose that the hypotheses of \Cref{ibsconvg} hold and that the Born and IBS converge.
	Let $\overline{\eta}$ denote the sum of the IBS and $\eta_1 = \mathcal{K}_{1}\left( \phi \right)$.
	Let $\mathcal{M} = \max \left\{ \lVert \eta \rVert_{\infty}, \lVert \tilde{\eta} \rVert_{\infty} \right\} $.
	Also assume that
	\begin{equation}\label{convgerror}
		\mathcal{M} < \frac{1}{2\kappa \mu}\left( 1 - \sqrt{\frac{\nu\kappa\mu \lVert \mathcal{K}_{1} \rVert}{1 + \nu\kappa\mu \lVert \mathcal{K}_{1} \rVert}}\right)
	.\end{equation}
Then the approximation error of reconstruction can be estimated as follows:
\begin{align*}
	\lVert \eta - \sum_{m=1}^{N} \mathcal{K}_{m}\left( \phi \right)  \rVert_{\infty}
	&\le M\left( \frac{\lVert \eta_1 \rVert_{\infty}}{r} \right)^{N + 1}\frac{1}{1-\frac{\lVert \eta_1 \rVert_{\infty}}{r}}\\
	&+ \left( 1-\frac{\nu\kappa\mu \lVert \mathcal{K}_{1} \rVert}{\left( 1-\mu \mathcal{M} \right)^2} + \nu\kappa\mu \lVert \mathcal{K}_{1} \rVert \right)^{-1} \lVert \left( I-\mathcal{K}_{1}K_1 \right) \eta \rVert_{\infty}
,\end{align*}
where
\[
M = \frac{2\mu\kappa}{\sqrt{16C + 1}}
.\] 
\end{thm}
\section{Appendix: Existence and Uniqueness}\label{sec2}

Our main tool in proving the existence and uniqueness of solutions is the Banach fixed point theorem.
Before stating the theorem, we give an important definition.

\begin{defn}
	Let $X$ be a normed vector space and let $B$ be a closed ball in $X$.
	A map $T\colon X\to X$ is called a \textit{contraction} on the closed ball $B$ if for any $f,g \in B$, we have
	\[
	\lVert T\left( f \right) -T\left( g \right) \rVert \le p \lVert f-g \rVert
	\]
	where $0< p < 1$.
\end{defn}
%TODO add the fixed point converges part
\begin{thm}[Banach]\label{thm1}
	Let $X$ be a Banach space and let $T\colon X\to X$ be a contraction on a closed ball $B\subseteq X$.
	If $T\left( B \right) \subseteq B$, then $T$ has a unique fixed point in $B$, that is, some $x \in B$ satisfies $T\left( x \right) = x$.
\end{thm}

Let $G\left( k; x, y \right) $ be the Green's function for the operator $\Delta u^{(1)} + k^2$
and
$G\left( 2k; x, y \right) $ be the Green's function for the operator $\Delta u^{(1)} + \left( 2k \right)^2$ as in \Cref{intro}.
Let
\[
	T\colon C\left( \overline{\Omega} \right)^2\to C\left( \overline{\Omega} \right)^2
\]
be given by
\begin{equation}\label{intro:2}
T\begin{pmatrix} u^{(1)}\\u^{(2)} \end{pmatrix}  = \begin{pmatrix} u_0^{(1)}\\u_0^{(2)} \end{pmatrix} +
\begin{pmatrix}
	-8\pi k^2 \int_{\Omega}^{} G\left(k; x,y \right) \eta\left( y \right) u^{(1)}\left( y \right) u^{(2)}\left( y \right)\dd{y}\\
	-4 \pi \left( 2k \right) ^2\int_{\Omega}^{} G\left( 2k; x,y \right) \eta\left( y \right) \left(u^{(1)}\right)^2\left( y \right)\dd{y}
\end{pmatrix}
.\end{equation}

If we can prove that \cref{intro:2} satisfies the hypothesis of \Cref{thm1} for some ball $B \subseteq C\left( \overline{\Omega} \right)^2$, then we will obtain a unique
$u= \begin{pmatrix} u^{(1)}\\u^{(2)} \end{pmatrix} \in C\left( \overline{\Omega} \right)^2$ that satisfies \cref{eq:1:1}, solving our PDEs.


We begin by making precise some of our statements in the introduction.
Define the norm $\lVert \cdot  \rVert$ on $C\left( \overline{\Omega} \right)^2$ to be
\[
\left\lVert \begin{pmatrix} u^{(1)}\\u^{(2)}\end{pmatrix} \right\rVert
\coloneqq
\lVert u^{(1)} \rVert_{C\left( \overline{\Omega} \right)}+
\lVert u^{(2)} \rVert_{C\left( \overline{\Omega} \right)}
.\]
Let
\begin{equation}\label{eq:1}
G\begin{pmatrix} u^{(1)}\\u^{(2)} \end{pmatrix}\coloneqq
\begin{pmatrix}
	-8\pi k^2 \int_{\Omega}^{} G\left(k; x,y \right) \eta\left( y \right) u^{(1)}\left( y \right) u^{(2)}\left( y \right)\dd{y}\\
	-4 \pi \left( 2k \right)^2\int_{\Omega}^{} G\left(2k; x,y \right) \eta\left( y \right) \left(u^{(1)}\right)^2\left( y \right)\dd{y}
\end{pmatrix}
.\end{equation}
Then, define the operator
\[
T\begin{pmatrix} u^{(1)}\\u^{(2)} \end{pmatrix}  = \begin{pmatrix} u_0^{(1)}\\u_0^{(2)} \end{pmatrix} + G \begin{pmatrix} u^{(1)}\\u^{(2)} \end{pmatrix}
.\]
\iffalse
\begin{align*}
	K_1 &\coloneqq-8\pi k^2 \int_{\Omega}^{} G\left( x,y \right) \eta\left( y \right) u^{(1)}\left( y \right) u^{(2)}\left( y \right)\dd{y}\\
	K_2 &\coloneqq-8\pi k^2 \int_{\Omega}^{} G\left( x,y \right) \eta\left( y \right) u^{(1)}^2\left( y \right)\dd{y}
\end{align*}
\fi
To bound the norm of $G$ in \cref{eq:1}, we define
\begin{align*}
	\mu_1 &\coloneqq 8\pi k^2 \sup_{x \in \Omega}\int_{\Omega}^{} \left| G\left(k; x,y \right) \right|  \dd{y}\\
	\mu_{2}&\coloneqq 16\pi k^2 \sup_{x \in \Omega}\int_{\Omega}^{} \left| G\left(2k; x,y \right) \right|  \dd{y}
\end{align*}
and let $\mu \coloneqq \max \left\{ \mu_1,\mu_2 \right\}$.
Now, we have
\iffalse
\begin{align*}
	\lVert K_1 \rVert &\le \mu\lVert \eta \rVert \lVert u^{(1)} u^{(2)} \rVert\le \mu\lVert \eta \rVert\lVert u^{(1)} \rVert\lVert u^{(2)} \rVert\\
	\lVert K_2 \rVert &\le \mu\lVert \eta \rVert \lVert u^{(1)}^2 \rVert \le \mu \lVert \eta \rVert\lVert u^{(1)} \rVert^2
.\end{align*}
\fi
\begin{align*}
	\left\lVert-8\pi k^2 \int_{\Omega}^{} G\left( x,y \right) \eta\left( y \right) u^{(1)}\left( y \right) u^{(2)}\left( y \right)\dd{y} \right\rVert
	&\le \mu\lVert \eta \rVert\lVert u^{(1)} \rVert\lVert u^{(2)} \rVert\\
	\left\lVert-4\pi \left( 2k \right)^2 \int_{\Omega}^{} G\left( x,y \right) \eta\left( y \right) \left(u^{(1)}\right)^2\left( y \right)\dd{y} \right\rVert
	&\le \mu \lVert \eta \rVert\lVert u^{(1)} \rVert^2
.\end{align*}

Remember that our goal is to show that $T$ satisfies \Cref{thm1}.
We begin by giving conditions under which $T$ is a contraction.
\begin{lem}\label{lem:1}
	Let $f = \begin{pmatrix} f^{(1)}\\f^{(2)} \end{pmatrix} $ and $g = \begin{pmatrix} g^{(1)}\\g^{(2)} \end{pmatrix}$.
	Let such that $\lVert f \rVert, \lVert g \rVert \le R$.
	Then,
	\[
	\lVert T f - T g \rVert
	\le q \lVert f - g \rVert
	\]
	where
	\[
	q = 2R\mu \lVert \eta \rVert
	.\]
\end{lem}
\begin{proof}
	Notice that
	\[
		T\left( f \right)  -T\left( g \right)
=
	\begin{pmatrix}
	-8\pi k^2 \int_{\Omega}^{} G\left( x,y \right) \eta\left( y \right) \left( f^{(1)}\left( y \right) f^{(2)}\left( y \right) -g^{(1)}\left( y \right) g^{(2)}\left( y \right)  \right) \dd{y}\\
	-8 \pi k^2\int_{\Omega}^{} G\left( x,y \right) \eta\left( y \right) \left(\left( f^{(1)}\right)^2\left( y \right) -\left(g^{(1)}\right)^2\left( y \right)  \right) \dd{y}
\end{pmatrix}
	\]
	so
	\begin{align*}
		\lVert T\left( f \right) -T\left( g \right)\rVert &\le \mu \lVert \eta \rVert \lVert f^{(1)}f^{(2)}-g^{(1)}g^{(2)} \rVert + \mu \lVert \eta \rVert \lVert \left(f^{(1)}\right)^2 - \left(g^{(1)}\right)^2 \rVert\\
		&\le \mu \lVert \eta \rVert \Big[ \lVert \left( f^{(1)}-g^{(1)} \right)f^{(2)} + \left( f^{(2)}-g^{(2)} \right)g^{(1)} \rVert + \lVert f^{(1)} + g^{(1)} \rVert\lVert f^{(1)}-g^{(1)} \rVert \Big] \\
		&\le \mu \lVert \eta \rVert \Big[ \lVert \left( f^{(1)}-g^{(1)} \right)\rVert\lVert f^{(2)} \rVert + \lVert \left( f^{(2)}-g^{(2)} \right)\rVert\lVert g^{(1)} \rVert + \left( \lVert f^{(1)} \rVert + \lVert g^{(1)} \rVert \right) \lVert f^{(1)}-g^{(1)} \rVert \Big] \\
		&= \mu \lVert \eta \rVert \Big[\left( \lVert f^{(1)} \rVert+ \lVert f^{(2)} \rVert+ \lVert g^{(1)} \rVert \right) \lVert f^{(1)}-g^{(1)} \rVert + \lVert g^{(1)} \rVert \lVert f^{(2)}-g^{(2)} \rVert  \Big]  \\
		&< \mu \lVert \eta \rVert \Big[2R \lVert f^{(1)}-g^{(1)} \rVert +R \lVert f^{(2)}-g^{(2)} \rVert  \Big] \\
		&< \mu \lVert \eta \rVert \Big[2R \lVert f^{(1)}-g^{(1)} \rVert +2R \lVert f^{(2)}-g^{(2)} \rVert  \Big]\\
		&= 2R\mu \lVert \eta \rVert \Big[\lVert f^{(1)}-g^{(1)} \rVert +\lVert f^{(2)}-g^{(2)} \rVert  \Big]\\
		&=2R \mu \lVert \eta \rVert \lVert f-g \rVert
	.\end{align*}
\end{proof}

The next lemma gives a ball that $T$ maps into itself (a ball $B\subseteq C\left( \overline{\Omega} \right)^2$ for which $T\left( B \right) \subseteq B$).
\begin{lem}\label{lem:2}
	Let $r > 0$ and let $B\left( u_0,r \right)$ be the ball of radius $r$ about $u_0 \coloneqq \begin{pmatrix} u_0^{(1)}\\u_0^{(2)} \end{pmatrix} $ in $C^{0}\left( \overline{\Omega} \right)^2$.
	Define $R\coloneqq \lVert u_0 \rVert+ r$. Then if
	\[
	\mu R^2 \lVert \eta \rVert< r
	,\]
	T maps $B\left( u_0,r \right)$ into itself.
\end{lem}
\begin{proof}
	Let $f = \begin{pmatrix} f^{(1)}\\f^{(2)} \end{pmatrix} \in B\left( u_0,r \right)$.
	Then,
	\begin{align*}
\lVert T\left( f \right) -u_0 \rVert&= \left\lVert G\begin{pmatrix} f^{(1)}\\f^{(2)} \end{pmatrix}\right\rVert \\
	&\le \mu\lVert \eta \rVert \lVert f^{(1)} \rVert\lVert f^{(2)} \rVert+ \mu \lVert \eta \rVert \lVert f^{(1)} \rVert^2\\
	&= \mu \lVert \eta \rVert\left( \lVert f^{(1)} \rVert\lVert f^{(2)} \rVert+ \lVert f^{(1)} \rVert^2 \right) \\
	&= \mu \lVert \eta \rVert \lVert f^{(1)} \rVert\left( \lVert f^{(2)} \rVert+ \lVert f^{(1)} \rVert \right) \\
	&< \mu \lVert \eta \rVert \lVert f^{(1)} \rVert R \\
	&< \mu \lVert \eta \rVert R^2
	.\end{align*}
\end{proof}
Since \Cref{lem:1} gives conditions for $T$ to be a contraction and  \Cref{lem:2} gives a ball that $T$ maps into itself,
if we can find $R > 0$ that satisfies both lemmas
simultaneously, we will have satisfied all hypothesis in \Cref{thm1} and solve our problem.
%TODO put thin in DM mode, it's the main result of the section. Maybe move the final result to the beginning.
\begin{prop}\label{lastprop}
	If
	\begin{equation}
		\lVert \eta \rVert \le \frac{1}{4\mu\lVert u_0 \rVert}
	,\end{equation}
	then $T$ has a unique fixed point in the ball $B\left( u_0, \lVert u_0 \rVert \right)$.
\end{prop}
\begin{proof}
	First, $r = \lVert u_0 \rVert\implies R = 2\lVert u_0 \rVert$.
	Thus,
	\begin{align*}
		\mu\lVert \eta \rVert 2R &\le \mu \lVert \eta \rVert 4 \lVert u_0 \rVert\\
	&< \mu 4 \lVert u_0 \rVert\cdot \frac{1}{\mu 4 \lVert u_0 \rVert}\\
	&= 1
	.\end{align*}
	And so we have satisfied \Cref{lem:1}.
	Now,
	\begin{align*}
	\mu \lVert \eta \rVert R^2 &\le \mu \lVert \eta \rVert \left( 2\lVert u_0 \rVert \right)^2\\
	&< \mu \frac{1}{4 \lVert u_0 \rVert}\cdot 4 \lVert u_0 \rVert^2\\
	 &= \lVert u_0 \rVert
	.\end{align*}
	Thus \Cref{lem:2} is also satisfied which implies that $T$ has a unique fix point on $B\left( u_0, \lVert u_0 \rVert \right)$.
\end{proof}

\begin{rem}
	Note that with \Cref{lastprop}, we have guaranteed the existence and uniqueness of solutions for any data (not just small input data).
\end{rem}

\section{Acknowledgments}
I would like to thank Shari Moskow for being such an excellent mentor. None of this work would be possible without her patience and guidance.
Much of my understanding of the subject comes from her detailed explanations--they have depend my appreciation for both PDEs and inverse problems.

\printbibliography

\end{document}
